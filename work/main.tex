\documentclass[a4paper,10pt,twoside]{report}

%ifdef PDF
%\usepackage[pdftex,colorlinks]{hyperref}
%endif

\usepackage[T1]{fontenc}
\usepackage[latin1]{inputenc}
\usepackage[ngerman]{babel}
%\usepackage[english]{babel}
\usepackage{ngerman,graphicx,float,latexsym,textcomp,longtable,verbatim,dsfont}

\setlength{\oddsidemargin}{1cm}
\setlength{\evensidemargin}{0cm}
\setlength{\textwidth}{15cm}

\pagestyle{headings}

%
%
% Includes
%
%%%%%%%%%%%
%\includeonly{title, intro, glossary}
%\includeonly{title, gisdb, app_gisdb, glossary}
%\includeonly{title, spatial, glossary}
%\includeonly{title, webservice, app_webservice, glossary}
%\includeonly{title, compreh, glossary}
%\includeonly{title, gisdb, webservice, app_gisdb, glossary}
%\includeonly{title, intro, gisdb, webservice, app_gisdb, app_webservice, glossary}

% Neue Umgebungen und Definitionen
%
%%%%%%%%%%%%%%%%%%%%%%%%%%%%%%%%%%%
%
% Hilfsdefinitionen
%%%%%%%%%%%%%%%%%%%
% der Betreuer soll den Abschnitt durchlesen
\newenvironment{lesen}%
{\vspace{5mm}\begin{sloppypar}
  \noindent\hrulefill\, BEGINN \,\hrulefill
  \end{sloppypar}
}%
{\begin{sloppypar}
  \noindent\hrulefill\, ENDE   \,\hrulefill
  \end{sloppypar}\vspace{5mm}
}

% der Betreuer soll den Abschnitt unter der Frage #1 durchlesen und
% kommentieren
\newenvironment{kommentar}[1]%
{\vspace{5mm}
  \begin{sloppypar}\noindent\hrulefill\, BEGINN Kommentar \,\hrulefill

  {\noindent\it #1}
  \end{sloppypar}
}%
{\begin{sloppypar}
  \noindent\hrulefill\, ENDE Kommentar   \,\hrulefill
  \end{sloppypar}\vspace{5mm}
}

% Frage an den Betreuer
\newcommand{\frage}[1]{\vspace{10mm}
\begin{sloppypar}{\bf Frage:} #1\end{sloppypar}
\vspace{10mm}}

% Bemerkung f�r den Betreuer
\newcommand{\bemerkung}[1]{\vspace{5mm}
\begin{sloppypar}{\bf Bemerkung:} #1\end{sloppypar}
\vspace{5mm}}

%%%%%%%%%%%%%%%%%%%%%%%%%%
% notwendige Definitionen
%%%%%%%%%%%%%%%%%%%%%%%%%%
\newtheorem{theorem}{Theorem}[chapter]
\newtheorem{lemma}{Lemma}[chapter]
\newtheorem{definition}{Definition}[chapter]
\newtheorem{beispiel}{Beispiel}[chapter]

% Beweis-Umgebung
\newenvironment{beweis}{\noindent {\bf Beweis:}}{\hfill$\Box$\smallskip}

% Numerierung von enumerate beginnt auf dem ersten Level mit Buchstaben
\renewcommand{\labelenumi}{(\alph{enumi})}
\renewcommand{\labelenumii}{(\roman{enumii})}

% Abk�rzungen
\newcommand{\lt}{$<$}
\newcommand{\gt}{$>$}
\newcommand{\ul}{\underline}
\newcommand{\ol}{\overline}
\newcommand{\zB}{z.\,B.\ }

% Trademarks
\newcommand{\multinet}{MultiNet{\small\texttrademark} }
\newcommand{\esri}{ESRI\raisebox{1ex}{\tiny\textregistered} }
\newcommand{\ibm}{IBM\raisebox{1ex}{\tiny\textregistered} }
\newcommand{\dbZ}{DB2\raisebox{1ex}{\tiny\textregistered} }

% Titel der Arbeit
\newcommand{\titel}{Tolle Teile Tauchen Tief}
%
% Beginn
%
%%%%%%%%%
\begin{document}

\begin{titlepage}
\hspace{20cm}
\vspace{-2cm}

\begin{figure}[H] \hspace{11cm}
\includegraphics[width=3.2 cm]{picture/husiegel}
\end{figure}

\begin{center}
  \vspace{0.5 cm}
  \huge{\bf Titel der Arbeit} \\ % Hier fuegen Sie den Titel Ihrer Arbeit ein.
  \vspace{1.5cm}
  \LARGE  ... \\ % Geben Sie anstelle der Punkte an, ob es sich um eine
                % Diplomarbeit, eine Masterarbeit oder eine Bachelorarbeit handelt.
  \vspace{1cm}
  \Large zur Erlangung des akademischen Grades \\
  ... \\ % Bitte tragen Sie hier anstelle der Punkte ein:
         % Diplominformatiker(in),
         % Bachelor of Arts (B. A.),
         % Bachelor of Science (B. Sc.),
         % Master of Education (M. Ed.) oder
         % Master of Science (M. Sc.).
  \vspace{2cm}
  {\large
    \bf{
      \scshape
      Humboldt-Universit\"at zu Berlin \\
      Mathematisch-Naturwissenschaftliche Fakult\"at II \\
      Institut f\"ur Informatik\\
    }
  } 
  % \normalfont
\end{center}
\vspace{5 cm}
{\large
  \begin{tabular}{llll}
    eingereicht von:    & ... && \\ % Bitte Vor- und Nachnamen anstelle der Punkte eintragen.
    geboren am:         & ... && \\
    in:                 & ... && \\
    &&&\\
    Gutachter(innen): & ... && \\
		      & ... && \\% Bitte Namen der Gutachter(innen) anstelle der Punkte eintragen
    &&&\\
    eingereicht am:     & \dots\dots & \hspace{3cm} verteidigt am: & \dots\dots \\ % Bitte lassen Sie
                                    % diese beiden Felder leer.
                                    % Loeschen Sie ggf. das letzte Feld, wenn
                                    % Sie Ihre Arbeit laut Pruefungsordnung nicht
                                    % verteidigen muessen.
  \end{tabular}
}

\end{titlepage}


% \thispagestyle{empty} % Seite hinter dem Titelblatt hat keine Seitennummer
% \cleardoublepage

\pagenumbering{roman}
\tableofcontents
% \listoffigures
% \listoftables

% \cleardoublepage

\chapter{Einleitung}
\pagenumbering{arabic}

Nach einem langen Spaziergang durch die Innenstadt von K�ln steht eine
Touristin vor dem Dom und m�chte mit den �ffentlichen Verkehrsmitteln wieder
zur�ck in ihr Hotel. Sie greift zum Mobiltelefon und stellt eine Verbindung
zum WAP-Portal ihres Mobilfunkanbieters her. Nach der Auswahl des Men�punktes
{\em �ffentliche Verkehrsmittel} �ffnet sich eine Eingabemaske, in dem
sie ihr Ziel, den Namen ihres Hotels, eingibt. Kurze Zeit sp�ter wird eine
Wegbeschreibung zur n�chstgelegenen Bushaltestelle sowie die Fahrverbindung
zu ihrem Hotel auf dem Display angezeigt.

Insgesamt realisiert das entworfene Szenario einen standortbezogenen Dienst
(Location Based Service). Unter Ausnutzung der Ortbarkeit eines eingeschalteten
Mobiltelefons lassen sich auf die momentane Position abgestimmte Dienste
und Informationen anbieten. F�r die Realisierung eines solchen Szenarios ist
ein Zusammenf�hren verschiedenster Datenquellen erforderlich.

\bigskip

\begin{figure}[htbp]
  \centering
  \includegraphics{picture/tradeoff}
  \caption{Tradeoffs in P2P systems}
  \label{fig:tradeoff}
\end{figure}

So beginnt zum Beispiel eine Studienarbeit. Nicht vergessen, hier in der Einleitung auch einen �berblick �ber die einzelnen Kapitel der Arbeit sowie deren Inhalt zu geben! So kann man z.B. mittels \texttt{ref} auf Verweise innerhalb des Dokumentes verweisen, wenn diese mit \texttt{label} vorher definiert wurden. Hier zum Beispiel ein Verweis auf das erste Kapitel im Anhang \ref{app:c1:s1} oder hier einer auf das nun folgende Bild \ref{fig:tradeoff}.

Ganz wichtig sind nat�rlich auch Zitate. Nat�rlich wird daf�r BibTeX verwendet, doch was man daf�r wissen muss, beschr�nkt sich auf relativ wenig: Anlegen und Pflegen einer .bib - Datei, am besten mit dem sehr guten Tool JabRef und zitieren im Text mit \texttt{cite}. So kann man dann beweisen, dass sich in einem Artikel \cite{Herschel2003} �ber Goya ge�u�ert wurde.

\include{inhalt}

% \include{appendix}

\bibliography{literatur}
\bibliographystyle{alpha}

% \cleardoublepage
\thispagestyle{empty}

%%%%%%%%%%%%%%%%%%%%%%%%%%%%%%%%%%%%%%%%%%%%%%%%%%%%%%%%%%%%%%%%%%%%%%%%%%%%%%%%%%%%%%%%%%%%%%%%%%%%
%% Selbststaendigkeitserklaerung
%%%%%%%%%%%%%%%%%%%%%%%%%%%%%%%%%%%%%%%%%%%%%%%%%%%%%%%%%%%%%%%%%%%%%%%%%%%%%%%%%%%%%%%%%%%%%%%%%%%%

{\parindent 0cm
%%%%%%%%%%%%%%%%%%%%%%%%%%deutsche Version%%%%%%%%%%%%%%%%%%%%%%%%%%%%%%
  
  \subsection*{Selbst\"andigkeitserkl\"arung}
  Ich erkl\"are hiermit, dass ich die vorliegende Arbeit selbst\"andig verfasst 
  und nur unter Verwendung der angegebenen Quellen und Hilfsmittel angefertigt habe. 
  Weiterhin erkl\"are ich, eine ...arbeit in diesem Studiengebiet erstmalig einzureichen.\\
%statt der Punkte Diplom, Master oder Bachelor angeben
  \vspace{3\baselineskip}
  
  Berlin, den \today \hspace{0.25\linewidth}\parbox{0.3\linewidth}{\dotfill}

%%%%%%%%%%%%%%%%%%%%%%%%%%englische Version%%%%%%%%%%%%%%%%%%%%%%%%%%%%%%%%%%%%%%%%%%%%%%%%%%%%%%%%%
\selectlanguage{english}
\subsection*{Statement of authorship}
I declare that I completed this thesis on my own and that information which has been 
directly or indirectly taken from other sources has been noted as such. Neither this 
nor a similar work has been presented to an examination committee.

  \vspace{3\baselineskip}
  
  Berlin, \today \hspace{0.25\linewidth}\parbox{0.3\linewidth}{\dotfill}
}



\end{document}
